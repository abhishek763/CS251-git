
\documentclass{article}

\title{Assignment 10:Git, Movie Reviews}
\author{Siddarth, Abhibhav, Abhishek}


\begin{document}
\maketitle
\section{Anand}
What an outstanding movie!! I have heard all the prior generation people rave about this movie, so, I decided to check this movie out myself. I only have faint memories of having watched parts of this movie from my mom's lap when she and dad were watching this in the theater. The other reason why I decided to check this out was a Super-bowl half-time debate on whether Amitabh was better than Rajesh Khanna. I could not participate in this debate for two reasons: first, I was eagerly awaiting another "wardrobe malfunction" for one of the cheerleaders and secondly, I only had memories of one Rajesh Khanna movie, Haathi Mere Saathi. I remember having enjoyed it very much as a child. But that alone was not enough to quantify anything. The more recent performances of AB were fresh in my mind, but after having seen this movie, I decided that Rajesh had a class of his own. His chirpy performance in this movie is really unparalleled! What an amazing performance! Amitabh, being more junior, has not equaled Rajesh, but has done his share very well. Thus, even after watching this movie, the debate will continue.

What's New? What can possibly be new in an old movie? Guess what? there is plenty for the younger generation to take away! There is no education in the movies these days, whatsoever, except perhaps bedroom or bar sequences. The inadequacies in the field of medicine are so nicely brought forward by this movie, which is certainly over 30 years old! Now, I can understand why Munnabhai MBBS was such a hit. It had so beautifully adapted from this movie to match the present generation. Kudos to Vidhu Vinod Chopra and Sanjay Dutt (Kamal Hassan too) for carrying this forward.

Noticeable: "BaaabuMushaai", the nick name for Amitabh, as heard from Rajesh throughout the movie, will ring in your ears even hours after you have seen the six letters "The End" on the screen. There really is no end to such people! Some sequences were simply amazing. Those that stood apart in my mind were the last scene, with a tape that had a significant pause in between; the moun-vrath guru, who so symbolically said that there is so much more than the decaying body to Anand's soul; then of course the Munirbhai sequences and the eventual backfiring of this strategy and so many more! The songs were so gentle and heart warming! The comic timing of Rajesh Khanna was simply amazing! Verdict Present day filmmakers really need to rework their brains and start thinking much much better! There is much more to Hindi cinema than just skin-show and catchy item number songs. This is a MUST-WATCH movie! I did not think so when others told me, but having experienced it myself, I believe them! I am gonna check out the other MUST-WATCH movies prescribed by the previous generation.

It is very rare for guys to shed tears after watching a movie, this movie does make you shed tears for Anand, the main protagonist played by the superstar of the 70's Rajesh Khanna.

The movie has everything going for it. Acting, direction, story, music, dialogues etc... everything is fabulous. It has drama, humor, emotions in dollops. It is a story a dying man who looks at life with a positive attitude and enjoys his time knowing fully well his disease is incurable and that he is going to die soon.

Rajesh Khanna as Anand is absolutely brilliant, this is his career-best performance , notwithstanding movies like Kati Patang, Roti, Aradhana, Amar Prem etc. You cannot think of any other actor in this role and to think Rajesh Khanna was not the original choice(Shashi Kapoor was). He makes you laugh and cry. He causes anxiety and goose bumps. Simply superb.

Amitabh Bachchan is fantastic as babumoshai( a name with which Raj Kapoor used to address the film's director Hrishikesh Mukherjee). He showed the world that the next superstar was coming, though he really "arrived" a couple of years later. The rest of the supporting cast is also brilliant be it Johhny Walker(stands out) or Ramesh Deo or Seema or Sumitra. Everyone is wonderful.

Music is the hallmark of all great hindi movies and this one has music ranking right up there, on the top. Be it "Kahin door jab din dhal jaye" or " Maine tere liye hi saath rang" or "Zindagi kaisi hai paheli".

Maverick composer Salil Chowdury comes up with an absolutely fantastic score and singers Mukhesh and Manna Dey do complete justice to his tunes.

About the director Hrishikesh Mukherjee, what can one say, he is one of the best directors ever in the Indian film history. A guy with a complete repertoire,a complete entertainer (though people consider other directors to be more entertaining, but real movie buffs will agree with me). All his movies, he has been directing movies since 1957 are worth a watch. Some are brilliant and others watchable. None of his movies can be rated as unwatchable(except maybe Jooth bole Kauwa kaate and Jhooti).

Simply put, this is one of the best Hindi movies ever made. 



\section{Ankur:The Seedling}
Ankur is a complex film that analyzes human behavior in general and heavily stresses characterization (though the story is not fictional). The story revolves around two characters, Lakshmi and Surya.

Lakshmi (Shabana Azmi) lives in a village with her husband Kishtayya (Sadhu Meher), a deaf-mute alcoholic potter who communicates using gestures. The couple are poor and belong to the lowly Dalit caste. Lakshmi attends a village festival and prays faithfully to the Goddess, stating that her only desire in life is to have a child.

Surya (Anant Nag), the son of the local village landlord, has just finished his studies in the nearby city of Hyderabad and arrives back home. Surya's father (Khader Ali Beg) has a mistress named Kaushalya with whom he has an illegitimate son named Pratap. Surya's father claims to have given Kaushalya "the best land in the village", a gift which serves as both a token of his affection and also keeps Kaushalya quiet and satisfied. Surya is forced by his father into a child marriage with the under aged Saru (Priya Tendulkar), and begins to feel extremely sexually frustrated due to the fact that they cannot have sex until Saru matures.

Surya reluctantly takes over the administrative responsibilities of his share of land in the village. Alone, he moves into a different, older house, and Lakshmi and Kishtayya are sent to act as his servants. Not long after his arrival, he begins to exert his authority by introducing a number of different laws and measures, many of which are controversial among the village people. Almost immediately, Surya starts to form an attraction towards Lakshmi, and gives her the task of cooking his meals and serving him tea. This does not sit well with the village priest, a man who traditionally delivers food to the landowner, though at a higher price than Lakshmi asks.

Surya also hires Kishtayya to ride his bullock cart and go on errands. The following day, he has Kishtayya collect fertilizer from the landlord's house. Surya then uses Kishtayya's absence to try to flirt with Lakshmi, but she fails to reciprocate. In the meantime, the villagers have begun to gossip, and many (most notably the overseer, Police Officer Patel Sheikh Chand) believe that Surya has already slept with Lakshmi, and will act in the same way that his father did - try to conceal the scandal by giving his mistress a vast plot of land.

Kishtayya is caught stealing toddy wine, after which he is publicly humiliated, and he decides to leave the village due to the embarrassment. In his absence, Surya and Lakshmi sleep together. Some while later, Saru arrives at the village, in order to live with her husband. Saru does not approve of Lakshmi's presence, partly because Lakshmi is a Dalit and partly because Saru has heard the villagers' rumors. The next morning, Lakshmi begins suffering from morning sickness, and Saru fires her, claiming that she is too sick to work.

Many days go by, and eventually Kishtayya returns, having cured himself of his alcoholism and made some money. Lakshmi is overwhelmed with a feeling of guilt, because she believes that she has betrayed her husband. On discovering Lakshmi's pregnancy, he salutes the village goddess at her temple, acknowledging that his wife's wish has been granted. He then decides to return to work and hopefully ride the bullock cart once again for Surya. Surya sees Kishtayya and mistakenly believes that Kishtayya is seeking revenge from him due to his infidelity with Lakshmi.

Surya orders three men to grab hold of Kishtayya and then proceeds to whip him with a rope used for lynching. The commotion attracts others, including Sheikh Chand and Pratap, to the scene, and Lakshmi rushes to defend her husband. She angrily curses Surya, then slowly returns home with Kishtayya. In the final scene, after the others have left, a young child throws a stone at Surya's glass window and runs away, and this stone represents the 'ankur'(seedling).

\end{document}
